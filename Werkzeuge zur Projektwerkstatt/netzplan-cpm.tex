\documentclass{beamer}

\mode<presentation>
\definecolor{hdaRot}{cmyk}{0,1,1,0.18}     % -5005
\definecolor{hdaBlauMed}{cmyk}{0.71,0.25,0,0}     % -1169
\definecolor{hdaGruenMed}{cmyk}{0.5,0.13,1,0.06}    % -2048

\setbeamercolor*{kred}{fg=hdaRot}
\setbeamercolor*{kblue}{fg=hdaBlauMed}
\setbeamercolor*{kgreen}{fg=hdaGruenMed}
\setbeamercolor*{kblack}{fg=black}
%\setbeamercolor{normal text}{fg=gray,bg=}
%\setbeamercolor{alerted text}{fg=black,bg=}
\setbeamercovered{transparent=25}
\usebeamercolor{normal text}
\mode<all>

\newcommand{\hlblue}{%
 \usebeamercolor[fg]{normal text}%
 \only{\usebeamercolor[fg]{kblue}}}

\newcommand{\hlgreen}{%
 \usebeamercolor[fg]{normal text}%
 \only{\usebeamercolor[fg]{kgreen}}}

\newcommand{\hlblack}{%
 \usebeamercolor[fg]{normal text}%
 \only{\usebeamercolor[fg]{kblack}}}

\newcommand{\hlred}{%
 \usebeamercolor[fg]{normal text}%
 \only{\usebeamercolor[fg]{kred}}}

\usepackage{tikz}

\usetikzlibrary{arrows,decorations.pathmorphing,backgrounds,positioning,fit,petri}%

\makeatletter
\tikzset{pics/named scope code/.style={code={\tikz@fig@mustbenamed%
  \begin{scope}[local bounding box/.expanded=\tikz@fig@name]#1\end{scope}%
}}}
\makeatother


\tikzset {
  pics/c4way/.style n args = {4}{
    named scope code = {
      \draw[line width=1pt]  (0,0) circle (1cm);
      \foreach \angle in {45, 135, 225, 315}{ \draw[line width=1pt] (\angle:0cm) -- (\angle:1cm); }
      \foreach \angle / \label in {90/#1, 180/#2, 0/#3, 270/#4}{ \draw (\angle:0.6cm) node{\textbf{\label}}; }
    }
  }
}

\begin{document}
\begin{frame}[t,shrink=65]
\begin{tikzpicture}
   [bend angle=45,
   line width=1pt,
   auto,
   pre/.style={<-,shorten <=1pt,>=stealth',semithick},
   post/.style={->,shorten >=1pt,>=stealth',semithick}]

   \pic (N01) at (0,4) {c4way=%
      {\hlblack<1>\visible<1->1}
      {\hlblue<12>\visible<12->0}
      {\hlgreen<30>\visible<30->0}
      {\hlred<32>\visible<32->0}
   };
   \pic (N02) at (4,0) {c4way=%
      {\hlblack<2>\visible<2->2}
      {\hlblue<12>\visible<12->1}
      {\hlgreen<13>\visible<30->2}
      {\hlred<31>\visible<31->1}
   };
   \pic (N03) at (4,4) {c4way=%
      {\hlblack<2>\visible<2->3}
      {\hlblue<13>\visible<13->2}
      {\hlgreen<29>\visible<29->2}
      {\hlred<32>\visible<32->0}
   };
   \draw<2-> [post] (N01) to node {Studie}
                      node[swap] {1}  (N02);
   \draw<2-> [post] (N01) to node {RQ}
                      node[swap] {2}  (N03);

   \draw<2-> [post] (N02) to node {}
                      node[swap] {}  (N03);

   \pic (N04) at (8,4) {c4way=%
      {\hlblack<3>\visible<3->4}
      {\hlblue<14>\visible<14->6}
      {\hlgreen<28>\visible<28->6}
      {\hlred<32>\visible<32->0}
   };
   \draw<3-> [post] (N03) to node {SE}
                      node[swap] {4}  (N04);

   \pic (N06) at (12,0) {c4way=%
      {\hlblack<4>\visible<4->6}
      {\hlblue<15>\visible<15->9}
      {\hlgreen<27>\visible<27->9}
      {\hlred<32>\visible<32->0}
   };
   \pic (N08) at (12,8) {c4way=%
      {\hlblack<5>\visible<5->8}
      {\hlblue<16>\visible<16->9}
      {\hlgreen<26>\visible<26->{11}}
      {\hlred<31>\visible<31->2}
   };
   \draw<4-> [post] (N04) to node {SW}
                      node[swap] {3}  (N06);
   \draw<5-> [post] (N04) to node {HW}
                      node[swap] {3}  (N08);

   \pic (N05) at (12,4) {c4way=%
      {\hlblack<1>\visible<6->5}
      {\hlblue<17>\visible<17->8}
      {\hlgreen<17>\visible<27->9}
      {\hlred<31>\visible<31->1}
   };
   \draw<6-> [post] (N04) to node {FM}
                      node[swap] {2}  (N05);
   \draw<6-> [post] (N05) to node {}
                      node[swap] {}  (N06);
   \draw<6-> [post] (N05) to node {}
                      node[swap] {}  (N08);

   \pic (N07) at (16,0) {c4way=%
      {\hlblack<1>\visible<7->7}
      {\hlblue<18>\visible<18->{15}}
      {\hlgreen<25>\visible<25->{15}}
      {\hlred<32>\visible<32->0}
   };
   \draw<7-> [post] (N06) to node {Pgm}
                      node[swap] {6}  (N07);

   \pic (N09) at (16,8) {c4way=%
      {\hlblack<1>\visible<8->9}
      {\hlblue<19>\visible<19->{14}}
      {\hlgreen<25>\visible<25->{16}}
      {\hlred<31>\visible<31->2}
   };
   \draw<8-> [post] (N08) to node {Proto}
                      node[swap] {5}  (N09);

   \pic (N10) at (20,4) {c4way=%
      {\hlblack<1>\visible<9->{10}}
      {\hlblue<20>\visible<20->{20}}
      {\hlgreen<24>\visible<24->{20}}
      {\hlred<32>\visible<32->0}
   };
   \draw<9-> [post] (N09) to node {HW-TST}
                      node[swap] {4}  (N10);
   \draw<9-> [post] (N07) to node {SW-TST}
                      node[swap] {5}  (N10);

   \pic (N11) at (24,4) {c4way=%
      {\hlblack<1>\visible<10->{11}}
      {\hlblue<21>\visible<21->{22}}
      {\hlgreen<23>\visible<23->{22}}
      {\hlred<32>\visible<32->0}
   };
   \draw<10-> [post] (N10) to node {Int}
                      node[swap] {2}  (N11);

   \pic (N12) at (28,4) {c4way=%
      {\hlblack<1>\visible<11->{12}}
      {\hlblue<22>\visible<22->{25}}
      {\hlgreen<22>\visible<22->{25}}
      {\hlred<32>\visible<32->0}
   };
   \draw<11-> [post] (N11) to node {Test}
                      node[swap] {3}  (N12);

 \end{tikzpicture}


{\large
    \begin{tabular}{l|lccl}
      \hline
      Nr & Tätigkeit        & Vorgänger & Aufwand & Kurz \\
      1  & Requirements     & -     & 2 & RQ \\
      2  & Studie           & -     & 1 & Studie \\
      3  & Systementwurf    & 1     & 4 & SE \\
      4  &   3              & 2     & - & \\
      5  & HW-Entwurf       & 3     & 3 & HW \\
      6  & Funktionsmuster  & 3     & 2 & FM \\
      7  & SW-Entwurf       & 3     & 3 & SW \\
      8  & Programmierung   & 7     & 6 & Pgm \\
      9  & 8                & 6     & - & \\
      10 & SW-Test          & 8     & 5 & SW-Test \\
      11 & Prototyp-Entwicklung & 5 & 5 & Proto \\
      12 & 11               & 6     & - & \\
      13 & HW-Test          & 11    & 4 & HW-Test \\
      14 & Integration      & 10; 13 & 2 & Int \\
      15 & System-Test      & 14    & 3 & Test \\
      \hline
        & \textbf{Gesamtaufwand} &   & \textbf{40} \\
    \end{tabular}
}



\begin{itemize}
{\huge
    \item<only@+> {Wir fange mit dem ersten Knoten an. Trage die 1 ein.}
    \item<only@+> {Suche alle Vorgänge ohne Vorgänger (siehe Liste). All diese gehen von unserem ersten Knoten aus. Schreibe die Tätigkeiten (in Kurzform)sowie Aufwand/ Dauer auf den Pfeil.}
    \item<only@+> {Füge alle möglichen Nachfolger zu den letzten Vorgängen hinzu.}
    \item \alert<+> {Bei einem Vorgang mit 2 oder mehr Nachfolgern, wird hier wie gewohnt Schritt 3 ausgeführt.}
}

\end{itemize}

\end{frame}


\end{document} 