\documentclass{beamer}

\usepackage{german}    %Block zum Setzen von Umlauten
\usepackage{lmodern}
\usepackage[utf8]{inputenc}

\mode<presentation>
\definecolor{hdaRot}{cmyk}{0,1,1,0.18}     % -5005
\definecolor{hdaBlauMed}{cmyk}{0.71,0.25,0,0}     % -1169
\definecolor{hdaGruenMed}{cmyk}{0.5,0.13,1,0.06}    % -2048

\setbeamercolor*{kred}{fg=hdaRot}
\setbeamercolor*{kblue}{fg=hdaBlauMed}
\setbeamercolor*{kgreen}{fg=hdaGruenMed}
\setbeamercolor*{kblack}{fg=black}
\setbeamerfont{block body alerted}{size=\huge}
\setbeamerfont{block title alerted}{size=\huge}
\setbeamercovered{transparent=25}
\usebeamercolor{normal text}
\mode<all>

\newcommand{\hlblue}{%
\usebeamercolor[fg]{normal text}%
\only{\usebeamercolor[fg]{kblue}}}

\newcommand{\hlgreen}{%
\usebeamercolor[fg]{normal text}%
\only{\usebeamercolor[fg]{kgreen}}}

\newcommand{\hlblack}{%
\usebeamercolor[fg]{normal text}%
\only{\usebeamercolor[fg]{kblack}}}

\newcommand{\hlred}{%
\usebeamercolor[fg]{normal text}%
\only{\usebeamercolor[fg]{kred}}}

\usepackage{tikz}
\usetikzlibrary{calc}

\usetikzlibrary{arrows,decorations.pathmorphing,backgrounds,positioning,fit,petri}%

\makeatletter
\tikzset{pics/named scope code/.style={code={\tikz@fig@mustbenamed%
  \begin{scope}[local bounding box/.expanded=\tikz@fig@name]#1\end{scope}%
}}}
\makeatother

\tikzset {
  pics/p8way/.style n args = {8}{
    named scope code = {
      \draw (-1.5,-0.75) rectangle  (1.5, 0.75);        %3cm lang & 1,5cm hoch
      \draw[line width=1pt]  (-1.5,-0.25) -- ( 1.5,-0.25);
      \draw[line width=1pt]  (-1.5, 0.25) -- ( 1.5, 0.25);
      \draw[line width=1pt]  (-0.5,-0.75) -- (-0.5, 0.75);
      \draw[line width=1pt]  ( 0.5,-0.75) -- ( 0.5, 0.25);

      \foreach \x / \y / \label in {
       -1/0.5/#1, 0.5/0.5/#2,
       -1/0/#3, 0/0/#4, 1/0/#5,
       -1/-0.5/#6, 0/-0.5/#7, 1/-0.5/#8
      }{\draw (\x,\y) node{\textbf{\label}};}
    }
  }
}

\begin{document}

\begin{frame}[t,shrink=58]
  \begin{tikzpicture}
    \pic (N01) at (0,0) {p8way=%
       {\hlblack<1->\visible<1->1}
       {\hlblack<1->\visible<1->{RQ}}
       {\hlblue<1>\visible<1->0}
       {\hlblack<1->\visible<1->2}
       {\hlblue<1>\visible<1->1}
       {\hlgreen<17>\visible<17->0}
       {\hlred<18>\visible<18->0}
       {\hlgreen<17>\visible<17->1}
    };
     \pic (N02) at (0,-3) {p8way=%
       {\hlblack<1->\visible<1->2}
       {\hlblack<1->\visible<1->{Studie}}
       {\hlblue<1>\visible<1->0}
       {\hlblack<1->\visible<1->1}
       {\hlblue<1>\visible<1->0}
       {\hlgreen<17>\visible<17->1}
       {\hlred<18>\visible<18->1}
       {\hlgreen<17>\visible<17->1}
    };
    \pic (N03) at (4,0) {p8way=%
       {\hlblack<2->\visible<2->3}
       {\hlblack<2->\visible<2->{SE}}
       {\hlblue<2>\visible<2->2}
       {\hlblack<2->\visible<2->4}
       {\hlblue<2>\visible<2->5}
       {\hlgreen<16>\visible<16->2}
       {\hlred<18>\visible<18->0}
       {\hlgreen<16>\visible<16->5}
    };
    \draw<2-> [post] (N01) to node {}
                      node[swap] {}  (N03);
    \draw<2-> [post] (N02) to node {}
                      node[swap] {}  (N03);
    \pic (N04) at (8,0) {p8way=%
       {\hlblack<3>\visible<3->5}
       {\hlblack<3>\visible<3->{FM}}
       {\hlblue<3>\visible<3->6}
       {\hlblack<3>\visible<3->2}
       {\hlblue<3>\visible<3->7}
       {\hlgreen<14>\visible<14->7}
       {\hlred<18>\visible<18->1}
       {\hlgreen<14>\visible<14->8}
    };
    \pic (N05) at (8,3) {p8way=%
       {\hlblack<3->\visible<3->4}
       {\hlblack<3>\visible<3->{HW}}
       {\hlblue<3>\visible<3->6}
       {\hlblack<3>\visible<3->3}
       {\hlblue<3>\visible<3->8}
       {\hlgreen<15>\visible<15->8}
       {\hlred<18>\visible<18->2}
       {\hlgreen<15>\visible<15->{10}}
    };
    \pic (N06) at (8,-3) {p8way=%
       {\hlblack<3>\visible<3->6}
       {\hlblack<3>\visible<3->{SW}}
       {\hlblue<3>\visible<3->6}
       {\hlblack<3>\visible<3->3}
       {\hlblue<3>\visible<3->8}
       {\hlgreen<15>\visible<15->6}
       {\hlred<18>\visible<18->0}
       {\hlgreen<15>\visible<15->8}
    };
    \draw<3-> [post] (N03) to node {}
                      node[swap] {}  (N04);
    \draw<3-> [post] (N03) to node {}
                      node[swap] {}  (N05);
    \draw<3-> [post] (N03) to node {}
                      node[swap] {}  (N06);
    \pic (N07) at (12,3) {p8way=%
       {\hlblack<4>\visible<4->7}
       {\hlblack<4>\visible<4->{Proto}}
       {\hlblue<4>\visible<4->9}
       {\hlblack<4>\visible<4->5}
       {\hlblue<4>\visible<4->{13}}
       {\hlgreen<12>\visible<12->{11}}
       {\hlred<18>\visible<18->2}
       {\hlgreen<12>\visible<12->{15}}
    };
    \pic (N08) at (12,-3) {p8way=%
       {\hlblack<4>\visible<4->8}
       {\hlblack<4>\visible<4->{Pgm}}
       {\hlblue<4>\visible<4->9}
       {\hlblack<4>\visible<4->6}
       {\hlblue<4>\visible<4->{14}}
       {\hlgreen<13>\visible<13->9}
       {\hlred<18>\visible<18->0}
       {\hlgreen<13>\visible<13->{14}}
    };
    \draw<4-> [post] (N04) to node {}
                      node[swap] {}  (N07);
    \draw<4-> [post] (N04) to node {}
                      node[swap] {}  (N08);
    \draw<4-> [post] (N05) to node {}
                      node[swap] {}  (N07);
    \draw<4-> [post] (N06) to node {}
                      node[swap] {}  (N08);
    \pic (N09) at (16,3) {p8way=%
       {\hlblack<5>\visible<5->9}
       {\hlblack<5>\visible<5->{HW-Test}}
       {\hlblue<5>\visible<5->{14}}
       {\hlblack<5>\visible<5->4}
       {\hlblue<5>\visible<5->{17}}
       {\hlgreen<11>\visible<11->{16}}
       {\hlred<18>\visible<18->2}
       {\hlgreen<11>\visible<11->{19}}
    };
    \draw<5-> [post] (N07) to node {}
                      node[swap] {}  (N09);
    \pic (N10) at (16,-3) {p8way=%
       {\hlblack<6>\visible<6->{10}}
       {\hlblack<6>\visible<6->{SW-Test}}
       {\hlblue<6>\visible<6->{15}}
       {\hlblack<6>\visible<6->5}
       {\hlblue<6>\visible<6->{19}}
       {\hlgreen<11>\visible<11->{15}}
       {\hlred<18>\visible<18->0}
       {\hlgreen<11>\visible<11->{19}}
    };
    \draw<6-> [post] (N08) to node {}
                      node[swap] {}  (N10);
    \pic (N11) at (20,0) {p8way=%
       {\hlblack<7>\visible<7->{11}}
       {\hlblack<7>\visible<7->{Int}}
       {\hlblue<7>\visible<7->{20}}
       {\hlblack<7>\visible<7->2}
       {\hlblue<7>\visible<7->{21}}
       {\hlgreen<10>\visible<10->{20}}
       {\hlred<18>\visible<18->0}
       {\hlgreen<10>\visible<10->{21}}
    };
    \draw<7-> [post] (N09) to node {}
                      node[swap] {}  (N11);
    \draw<7-> [post] (N10) to node {}
                      node[swap] {}  (N11);
    \pic (N12) at (24,0) {p8way=%
       {\hlblack<8>\visible<8->{12}}
       {\hlblack<8>\visible<8->{Test}}
       {\hlblue<8>\visible<8->{22}}
       {\hlblack<8>\visible<8->3}
       {\hlblue<8>\visible<8->{24}}
       {\hlgreen<9>\visible<9->{22}}
       {\hlred<18>\visible<18->0}
       {\hlgreen<9>\visible<9->{24}}
    };
    \draw<8-> [post] (N11) to node {}
                      node[swap] {}  (N12);

    %Einzeichnen des kritischen Pfades
    \only<18>{\draw[red, ultra thick] [post] (N11) to node {}
                      node[swap] {}  (N12);}
    \only<18>{\draw[red, ultra thick] [post] (N10) to node {}
                      node[swap] {}  (N11);}
    \only<18>{\draw[red, ultra thick] [post] (N08) to node {}
                      node[swap] {}  (N10);}
    \only<18>{\draw[red, ultra thick] [post] (N06) to node {}
                      node[swap] {}  (N08);}
    \only<18>{\draw[red, ultra thick] [post] (N03) to node {}
                      node[swap] {}  (N06);}
    \only<18>{\draw[red, ultra thick] [post] (N01) to node {}
                      node[swap] {}  (N03);}

  \end{tikzpicture}

\begin{columns}
\column[t]{.50\textwidth}

{\Large

\par\vspace{2cm}\noindent       %Abstand zur Grafik 2cm

    \begin{tabular}{l|lccl}
      \hline
      Nr & Tätigkeit        & Vorgänger & Aufwand & Kurz \\
      1  & Requirements     & -     & 2 & RQ \\
      2  & Studie           & -     & 1 & Studie \\
      3  & Systementwurf    & 1     & 4 & SE \\
      4  & HW-Entwurf       & 3     & 3 & HW \\
      5  & Funktionsmuster  & 3     & 2 & FM \\
      6  & SW-Entwurf       & 3     & 3 & SW \\
      7  & Programmierung   & 6     & 6 & Pgm \\
      8 & SW-Test          & 7     & 5 & SW-Test \\
      9 & Prototyp-Entwicklung & 4 & 5 & Proto \\
      10 & HW-Test          & 9    & 4 & HW-Test \\
      11 & Integration      & 8; 10 & 2 & Int \\
      12 & System-Test      & 11    & 3 & Test \\
      \hline
        & \textbf{Gesamtaufwand} &   & \textbf{40} \\
    \end{tabular}
}

\column[t]{.50\textwidth}
\par\vspace{1 cm}\noindent %Abstand
\begin{itemize}

{\LARGE
    \item<only@+> {Alle Vorgänge ohne Vorgänger(siehe Liste) werden als Startknoten gewählt. Die Tätigkeit wird oben rechts in die größere Spalte geschrieben und der Aufwand darunter in die Mitte. Der Netzplan beginnt immer bei einem FAZ(Frühester Anfangszeitpunkt) von 0(linke mittlere Spalte). FEZ(Frühester Endzeitpunkt) = FAZ + T - 1(rechte mittlere Spalte).}

    \item<only@+> {Suche Tätigkeiten, die die Startknoten als Vorgänger haben(Liste) -{\textgreater} diese sind die nächsten ausgehenden Knoten. Bei mehreren Vorgängern wird der MAX(FAZ) als neuer FAZ genommen, jedoch ist zu beachten, dass bei dem Übergang von einem Knoten zum anderen der FAZ+1 genommen. Rest wird wie gewohnt hinzugefügt.}
    \item<only@+> {Schritte wie gewohnt.}
    \item<only@+> {Siehe Schritt 2(mehrere Knoten gehen über in einen Knoten, somit ist MAX(FAZ) gefragt).}
    \item<only@+> {Schritt wie gewohnt.}
    \item<only@+> {Schritt wie gewohnt.}
    \item<only@+> {Schritt wie gewohnt.}
    \item<only@+> {Schritt wie gewohnt.}
    \item<only@+> {Nun wird der letzte FEZ automatisch als SEZ(rechte untere Spalte) übernommen. SAZ berechnet sich folgendermaßen: SAZ = SEZ - T + 1 (linke untere Spalte)}
    \item<only@+> {Bei einem Übergang von einem Knoten zum anderen wird der davorige SEZ-1 genommen. Rest verläuft wie gewohnt.}
    \item<only@+> {Schritt wie gewohnt.}
    \item<only@+> {Schritt wie gewohnt.}
    \item<only@+> {Schritt wie gewohnt.}
    \item<only@+> {Hierbei wird geschaut, welcher SEZ kleiner ist. Dieser wird als SAZ - 1 übernommen.}
    \item<only@+> {Schritt wie gewohnt.}
    \item<only@+> {Schritt wie gewohnt.}
    \item<only@+> {Die Differenz aus SEZ-FEZ ergibt die Pufferzeit. Zeichne den kritischen Pfad ein (rot markiert).}
    \item \alert<+> {}
}

\end{itemize}

\par\vspace{0.1cm}\noindent %Abstand

\only<1>{\begin{alertblock}{Merke!}
    Nummerierung der Knoten hat nichts mit Nummerierung der Vorgänge zu tun. Knotennummer wird beliebig gesetzt.
\end{alertblock}
}

\only<18>{\begin{alertblock}{Merke!}
   Alle Vorgänge mit einer Pufferzeit von 0 gehören zu dem kritischen Pfad.
\end{alertblock}
}

\end{columns}
\end{frame}
\end{document} 