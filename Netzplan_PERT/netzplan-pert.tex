\documentclass{beamer}

\usepackage{german}  %Block zum Setzen von Umlauten
\usepackage{lmodern}
\usepackage[utf8]{inputenc}

\mode<presentation>
\definecolor{hdaRot}{cmyk}{0,1,1,0.18}     % -5005
\definecolor{hdaBlauMed}{cmyk}{0.71,0.25,0,0}     % -1169
\definecolor{hdaGruenMed}{cmyk}{0.5,0.13,1,0.06}    % -2048

\setbeamercolor*{kred}{fg=hdaRot}
\setbeamercolor*{kblue}{fg=hdaBlauMed}
\setbeamercolor*{kgreen}{fg=hdaGruenMed}
\setbeamercolor*{kblack}{fg=black}
\setbeamerfont{block body alerted}{size=\huge}
\setbeamerfont{block title alerted}{size=\huge}
\setbeamercovered{transparent=25}
\usebeamercolor{normal text}
\mode<all>

\newcommand{\hlblue}{%
 \usebeamercolor[fg]{normal text}%
 \only{\usebeamercolor[fg]{kblue}}}

\newcommand{\hlgreen}{%
 \usebeamercolor[fg]{normal text}%
 \only{\usebeamercolor[fg]{kgreen}}}

\newcommand{\hlblack}{%
 \usebeamercolor[fg]{normal text}%
 \only{\usebeamercolor[fg]{kblack}}}

\newcommand{\hlred}{%
 \usebeamercolor[fg]{normal text}%
 \only{\usebeamercolor[fg]{kred}}}

\usepackage{tikz}

\usetikzlibrary{arrows,decorations.pathmorphing,backgrounds,positioning,fit,petri}%

\makeatletter
\tikzset{pics/named scope code/.style={code={\tikz@fig@mustbenamed%
  \begin{scope}[local bounding box/.expanded=\tikz@fig@name]#1\end{scope}%
}}}
\makeatother


\tikzset {
  pics/p2way/.style n args = {4}{
    named scope code = {
      \draw[line width=1pt]  (0,0) circle (1cm);
      \draw[line width=1pt] (210:1cm) -- (330:1cm);
      \foreach \angle / \radius / \label in {90/3mm/#1, 270/7mm/#2}{ \draw (\angle:\radius) node{\textbf{\label}}; }
       \foreach \angle / \label in {90/#3, 270/#4}{ \draw (\angle:1.3cm) node{\textbf{\label}}; }
    }
  }
}

\begin{document}

%\begin{frame}[t,shrink=65]
\begin{frame}[t,shrink=65]{Vorganspfeil Netzplan}
\begin{tikzpicture}
   [bend angle=45,
   line width=1pt,
   auto,
   pre/.style={<-,shorten <=1pt,>=stealth',semithick},
   post/.style={->,shorten >=1pt,>=stealth',semithick}]

  \pic (N00) at (0,-4) {p2way=%
    {\hlblack<1>\visible<1->{Start}}
    {\hlblack<1>\visible<1->0}
    {\hlblue<9>\visible<9->{T$_{F}$ = 0}}
    {\hlgreen<27>\visible<27->{T$_{S}$ = 0}}
  };

  \pic (N01) at (4,0) {p2way=%
    {\hlblack<1>\visible<1->{RQ}}
    {\hlblack<1>\visible<1->1}
    {\hlblue<10>\visible<10->{T$_{F}$ = 2,0}}
    {\hlgreen<26>\visible<26->{T$_{S}$ = 2,0}}
  };

  \pic (N02) at (4,-8) {p2way=%
  	{\hlblack<2>\visible<1->{Studie}}
  	{\hlblack<2>\visible<1->2}
  	{\hlblue<10>\visible<10->{T$_{F}$ = 1,5}}
  	{\hlgreen<26>\visible<26->{T$_{S}$ = 1,5}}
  };

  \pic (N03) at (4,-4) {p2way=%
  	{\hlblack<3>\visible<3->{SE St.}}
  	{\hlblack<3>\visible<3->{3+}}
  	{\hlblue<11>\visible<11->{T$_{F}$ = 2,0}}
  	{\hlgreen<25>\visible<25->{T$_{S}$ = 2,0 }}
  };

  \draw<2-> [post] (N00) to node {1/2/3} node[swap] {2,0} (N01);
  \draw<2-> [post] (N00) to node {1/1/4} node[swap] {1,5} (N02);

  \draw<3-> [post] (N01) to node { } node[swap] { } (N03);
  \draw<3-> [post] (N02) to node { } node[swap] { } (N03);

  \pic (N04) at (8,-4) {p2way=%
  	{\hlblack<4>\visible<4->{SE}}
  	{\hlblack<4>\visible<4->3}
  	{\hlblue<12>\visible<12->{T$_{F}$ = 6,0 }}
  	{\hlgreen<24>\visible<24->{T$_{S}$ = 6,0 }}
  };

  \draw<4-> [post] (N03) to node {3/4/5} node[swap] {4,0}  (N04);

  \pic (N05) at (12,0) {p2way=%
  	{\hlblack<4>\visible<4->{HW}}
  	{\hlblack<4>\visible<4->4}
  	{\hlblue<13>\visible<13->{T$_{F}$ = 9 }}
  	{\hlgreen<23>\visible<23->{T$_{S}$ = 13,5 }}
  };

  \pic (N06) at (12,-4) {p2way=%
  	{\hlblack<4>\visible<4->{FM}}
  	{\hlblack<4>\visible<4->5}
  	{\hlblue<13>\visible<13->{T$_{F}$ = 8 }}
  	{\hlgreen<23>\visible<23->{T$_{S}$ = 9,5 }}
  };

  \pic (N07) at (12,-8) {p2way=%
  	{\hlblack<4>\visible<4->{SW}}
  	{\hlblack<4>\visible<4->6}
  	{\hlblue<13>\visible<13->{T$_{F}$ = 9,5}}
  	{\hlgreen<23>\visible<23->{T$_{S}$ = 9,5}}
  };

  \draw<4-> [post] (N04) to node {2/3/4} node[swap] {3,0}  (N05);
  \draw<4-> [post] (N04) to node {1/2/3} node[swap] {2,0}  (N06);
  \draw<4-> [post] (N04) to node {2/3/7} node[swap] {3,5}  (N07);

  \draw<5-> [post] (N06) to node { } node[swap] { }  (N05);
  \draw<5-> [post] (N06) to node { } node[swap] { }  (N07);

  \pic (N08) at (16,0) {p2way=%
  	{\hlblack<6>\visible<5->{Proto}}
  	{\hlblack<6>\visible<5->9}
  	{\hlblue<14>\visible<14->{T$_{F}$ = 14}}
  	{\hlgreen<22>\visible<22->{T$_{S}$ = 16,5}}
  };

   \pic (N09) at (16,-8) {p2way=%
  	{\hlblack<6>\visible<5->{Pgm}}
  	{\hlblack<6>\visible<5->7}
  	{\hlblue<14>\visible<14->{T$_{F}$ = 15,5}}
  	{\hlgreen<22>\visible<22->{T$_{S}$ = 15,5}}
  };

  \draw<6-> [post] (N05) to node {4/5/6}  node[swap] {5}  (N08);
  \draw<6-> [post] (N07) to node {4/6/8}  node[swap] {6}  (N09);

  \pic (N10) at (20,0) {p2way=%
  	{\hlblack<6>\visible<6->{HWT}}
  	{\hlblack<6>\visible<6->{10}}
  	{\hlblue<15>\visible<15->{T$_{F}$ = 18}}
  	{\hlgreen<21>\visible<21->{T$_{S}$ = 20,5}}
  };

   \pic (N11) at (20,-8) {p2way=%
  	{\hlblack<6>\visible<6->{SWT}}
  	{\hlblack<6>\visible<6->8}
  	{\hlblue<15>\visible<15->{T$_{F}$ = 20,5}}
  	{\hlgreen<21>\visible<21->{T$_{S}$ = 20,5}}
  };

  \draw<6-> [post] (N08) to node {3/4/5} node[swap] {4}  (N10);
  \draw<6-> [post] (N09) to node {4/5/6} node[swap] {5}  (N11);

  \pic (N12) at (20,-4) {p2way=%
    {\hlblack<7>\visible<7->{IntSt.}}
    {\hlblack<7>\visible<7->{11+}}
    {\hlblue<16>\visible<16->{T$_{F}$ = 20,5}}
    {\hlgreen<20>\visible<20->{T$_{S}$ = 20,5}}
  };

  \pic (N13) at (24,-4) {p2way=%
    {\hlblack<7>\visible<7->{Int}}
    {\hlblack<7>\visible<7->{11}}
    {\hlblue<17>\visible<17->{T$_{F}$ = 22,5}}
    {\hlgreen<19>\visible<19->{T$_{S}$ = 22,5}}
  };

  \pic (N14) at (28,-4) {p2way=%
  	{\hlblack<8>\visible<8->{Test}}
  	{\hlblack<8>\visible<8->{12}}
  	{\hlblue<18>\visible<18->{T$_{F}$ = 25,5}}
  	{\hlgreen<18>\visible<18->{T$_{S}$ = 25,5}}
  };

  \draw<7-> [post] (N10) to node { } node[swap] { } (N12);
  \draw<7-> [post] (N11) to node { } node[swap] { } (N12);

  \draw<8-> [post] (N12) to node {1/2/3}  node[swap] {2} (N13);
  \draw<8-> [post] (N13) to node {2/3/4}  node[swap] {3} (N14);

%Einzeichnen des kritischen Pfades
  \only<28>{\draw[red, ultra thick] [post] (N00)
    to node {} node[swap] {} (N01);}
  \only<28>{\draw[red, ultra thick] [post] (N01)
    to node {} node[swap] {} (N03);}
  \only<28>{\draw[red, ultra thick] [post] (N03)
   to node {} node[swap] {}  (N04);}
  \only<28>{\draw[red, ultra thick] [post] (N04)
   to node {} node[swap] {}  (N07);}
  \only<28>{\draw[red, ultra thick] [post] (N07)
   to node {} node[swap] {}  (N09);}
  \only<28>{\draw[red, ultra thick] [post] (N09)
   to node {} node[swap] {}  (N11);}
  \only<28>{\draw[red, ultra thick] [post] (N11)
   to node {} node[swap] {}  (N12);}
  \only<28>{\draw[red, ultra thick] [post] (N12)
   to node {} node[swap] {}  (N13);}
  \only<28>{\draw[red, ultra thick] [post] (N13)
   to node {} node[swap] {}  (N14);}
\end{tikzpicture}

\begin{columns}
  \column[t]{.50\textwidth}
    \par\vspace{2cm}\noindent       %Abstand zur Grafik 2cm
    {\Large
    \begin{tabular}{l|lccccrl}
      Nr  & Tätigkeit       & Vorgänger & T$_{o}$ & T$_{w}$ & T$_{p}$ & T$_{m}$ & Kurz \\
    \hline
       1  & Requirements          & -     & 1 & 2 & 3   & 2,0 & RQ  \\
       2  & Studie                & -     & 1 & 1 & 4   & 1,5 & Studie\\
       3  & Sytsementwurf         & 1; 2  & 3 & 4 & 5   & 4,0 & SE  \\
       4  & HW-Entwurf            & 3     & 2 & 3 & 4   & 3,0 & HW  \\
       5  & Funktionsmuster       & 3     & 1 & 2 & 3   & 2,0 & FM  \\
       6  & SW-Entwurf            & 3     & 2 & 3 & 7   & 3,5 & SW  \\
       7  & Programmierung        & 6     & 4 & 6 & 8   & 6,0 & Pgm \\
       8  & SW-Test               & 7     & 4 & 5 & 6   & 5,0 & SWT \\
       9  & Prototyp-Entwicklung  & 4     & 4 & 5 & 6   & 5,0 & Proto \\
      10  & HW-Test               & 9     & 3 & 4 & 5   & 4,0 & HWT \\
      11  & Integration           & 8; 10 & 1 & 2 & 3   & 2,0 & Int \\
      12  & Sytsem-Test           & 11    & 2 & 3 & 4   & 3,0 & Test \\
    \hline
          & \textbf{Gesamtaufwand} & & & \textbf{40} & & \textbf{41,0} \\
    \end{tabular}
    }
  \column[t]{.45\textwidth}
    \par\vspace{1em}\noindent       %Abstand zur Grafik 2cm
    {\huge
    \begin{itemize}
      \item<only@1-8> \textbf{Tätigkeiten und Ereignisse}
      \item<only@9-17> \textbf{Vorwärtsrechnung}
      \item<only@18-26> \textbf{Rückwärtsrechnung}
      \item<only@27-> \textbf{Kritischer Pfad}

      \item<only@1> Suche alle Tätigkeiten ohne Vorgänger ('-'). Alle diese Tätigkeiten beginnen im ersten Knoten (Start).

      \item<only@2-8> Schreibe die Tätigkeit (in Kurzform), sowie die fortlaufende Nummer in den Knoten, wo eine Tätigkeit abgeschlossen wird (Ereignis).

      \item<only@2-8> Füge den jew. Pfeil zum Nachfolger hinzu und schreibe T$_o$, T$_w$ und T$_p$ über den Pfeil. Berechne anschließend T$_m$ und schreibe diesen Wert unter den Pfeil.

      \item<only@2-8> T$_m$ = $\frac{ T_{o} + 4T_{w} + T_{p}}{6}$

      \item<only@3,7> Hat eine Tätigkeit \textbf{mehrere Vorgänger}, füge ein Startergeignis ein, an dem diese synchron starten. Verbinde den Vorgänger mit dem Startknoten. Diese Pfeile haben keine Dauer.

      \item<only@5> Hat eine Tätigkeit \textbf{keinen Nachfolger}, so ist diese ggf. mit anderen Ereignissen zu Synchronisieren. Diese Pfeile haben keine Tätigkeit, also keine Dauer.

      \item<only@9> Der Netzplan beginnt bei T$_F$ = 0. Schreibe dies über den Startknoten.

      \item<only@10-17> Bei der Vorwärtsberechnung wird T$_{m}$ zu T$_{F}$ addiert. Schreibe dies über den Knoten an dem die Tätigkeit endet.

      \item<only@11,12,14> Bei der Vorwärtsberechnung wird mit dem größeren Wert weitergearbeitet, wenn ein Knoten mehrere Vorgänger hat.

      \item<only@17> Am letzten Knoten wird der Wert für T$_{S}$ von  T$_{F}$ übernommen und unter den Knoten geschrieben.

      \item<only@18-26> Nun folgt die Rückwärtsberechnung. Hierbei wird T$_{m}$ von T$_{S}$ abgezogen und unter den vorherigen Knoten geschrieben.

      \item<only@21,24> Wenn ein Knoten \textbf{mehrere Nachfolger} hat, wird mit dem kleinsten Wert der T$_{S} -$ T$_{m}$  weitergearbeitet.

      \item<only@27-> Berechne die Pufferwerte P = T$_{S}$ - T$_{F}$. Alle mit  $P=0$ liegen auf dem kritischen Pfad.

      \item<only@28> Zeichne den kritischen Pfad ein (hier: rot markiert).
%      \item \alert<+> {}
    \end{itemize}
    }
  \end{columns}
\end{frame}

\end{document}
