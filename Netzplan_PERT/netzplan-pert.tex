\documentclass{beamer}
\usepackage{german} %Block zum Setzen von Umlauten
\usepackage{lmodern}
\usepackage[utf8]{inputenc}
\mode<presentation>
\definecolor{hdaRot}{cmyk}{0,1,1,0.18} % -5005
\definecolor{hdaBlauMed}{cmyk}{0.71,0.25,0,0} % -1169
\definecolor{hdaGruenMed}{cmyk}{0.5,0.13,1,0.06} % -2048
\setbeamercolor*{kred}{fg=hdaRot}
\setbeamercolor*{kblue}{fg=hdaBlauMed}
\setbeamercolor*{kgreen}{fg=hdaGruenMed}
\setbeamercolor*{kblack}{fg=black}
\setbeamerfont{block body alerted}{size=\huge}
\setbeamerfont{block title alerted}{size=\huge}
\setbeamercovered{transparent=25}
\usebeamercolor{normal text}
\mode<all>
\newcommand{\hlblue}{%
\usebeamercolor[fg]{normal text}%
\only{\usebeamercolor[fg]{kblue}}}
\newcommand{\hlgreen}{%
\usebeamercolor[fg]{normal text}%
\only{\usebeamercolor[fg]{kgreen}}}
\newcommand{\hlblack}{%
\usebeamercolor[fg]{normal text}%
\only{\usebeamercolor[fg]{kblack}}}
\newcommand{\hlred}{%
\usebeamercolor[fg]{normal text}%
\only{\usebeamercolor[fg]{kred}}}
\usepackage{tikz}
\usetikzlibrary{arrows,decorations.pathmorphing,backgrounds,positioning,fit,petri}%
\makeatletter
\tikzset{pics/named scope code/.style={code={\tikz@fig@mustbenamed%
\begin{scope}[local bounding box/.expanded=\tikz@fig@name]#1\end{scope}%
}}}
\makeatother
\tikzset {
pics/p2way/.style n args = {2}{
named scope code = {
\draw[line width=1pt] (0,0) circle (1cm);
\foreach \angle in {0, 180}{ \draw[line width=1pt] (\angle:0cm) -- (\angle:1cm); }
\foreach \angle / \label in {90/#1, 270/#2}{ \draw (\angle:0.6cm) node{\textbf{\label}}; }
}
}
}
\begin{document}
\begin{frame}[t,shrink=65]
\begin{tikzpicture}
[bend angle=45,
line width=1pt,
auto,
pre/.style={<-,shorten <=1pt,>=stealth',semithick},
post/.style={->,shorten >=1pt,>=stealth',semithick}]
\pic (N01) at (0,-4) {p2way=%
{\hlblack<1>\visible<1->RQ}
{\hlblack<1>\visible<1->1}
};
\pic (N02) at (2,-8) {p2way=%
{\hlblack<1>\visible<1->Studie}
{\hlblack<1>\visible<1->2}
};
\pic (N03) at (4,-4) {p2way=%
{\hlblack<1>\visible<1->SE}
{\hlblack<1>\visible<1->3}
};
\draw<2-> [post] (N01) to node {1/1/2}
node[swap] {1} (N02);
\draw<2-> [post] (N01) to node {1/2/3}
node[swap] {2} (N03);
\draw<2-> [post] (N02) to node {}
node[swap] {} (N03);
\pic (N04) at (8,0) {p2way=%
{\hlblack<1>\visible<1->HW}
{\hlblack<1>\visible<1->5}
};
\draw<3-> [post] (N03) to node {3/4/5}
node[swap] {4} (N04);
\draw<3-> [post] (N03) to node {1,5/2/2,5}
node[swap] {2} (N05);
\draw<3-> [post] (N03) to node {2,5/3/4}
node[swap] {3} (N07);
\pic (N05) at (8,-4) {p2way=%
{\hlblack<1>\visible<1->FM}
{\hlblack<1>\visible<1->6}
};
\pic (N13) at (12,-2.5) {p2way=%
{\hlblack<1>\visible<1->8}
{\hlblack<1>\visible<1->9}
};
\pic (N14) at (12,-5.5) {p2way=%
{\hlblack<1>\visible<1->11}
{\hlblack<1>\visible<1->12}
};
\pic (N06) at (12,0) {p2way=%
{\hlblack<1>\visible<1->Proto}
{\hlblack<1>\visible<1->11}
};
\pic (N07) at (8,-8) {p2way=%
{\hlblack<1>\visible<1->SW}
{\hlblack<1>\visible<1->7}
};
\pic (N08) at (12,-8) {p2way=%
{\hlblack<1>\visible<1->Pgm}
{\hlblack<1>\visible<1->8}
};
\draw<4-> [post] (N04) to node {4/5/6}
node[swap] {5} (N06);
\draw<4-> [post] (N07) to node {5/6/7,5}
node[swap] {6} (N08);
\pic (N09) at (16,0) {p2way=%
{\hlblack<1>\visible<1->HWT}
{\hlblack<1>\visible<1->1}
};
\pic (N10) at (16,-8) {p2way=%
{\hlblack<1>\visible<1->SWT}
{\hlblack<1>\visible<1->10}
};
\draw<5-> [post] (N06) to node {3/4/5}
node[swap] {4} (N09);
\draw<5-> [post] (N08) to node {4/5/6}
node[swap] {5} (N10);
\pic (N11) at (20,-4) {p2way=%
{\hlblack<1>\visible<1->Int}
{\hlblack<1>\visible<1->14}
};
\pic (N12) at (24,-4) {p2way=%
{\hlblack<1>\visible<1->Test}
{\hlblack<1>\visible<1->15}
};
\draw<6-> [post] (N09) to node {1/2/3}
node[swap] {2} (N11);
\draw<6-> [post] (N10) to node {1/2/3}
node[swap] {2} (N11);
\draw<7-> [post] (N11) to node {2/3/4}
node[swap] {3} (N12);
%Einzeichnen des kritischen Pfades
\only<23>{\draw<14->[red, ultra thick] [post] (N01) to node {}
node[swap] {} (N03);}
\only<23>{\draw<14->[red, ultra thick] [post] (N03) to node {}
node[swap] {} (N07);}
\only<23>{\draw<14->[red, ultra thick] [post] (N07) to node {}
node[swap] {} (N08);}
\only<23>{\draw<14->[red, ultra thick] [post] (N08) to node {}
node[swap] {} (N10);}
\only<23>{\draw<14->[red, ultra thick] [post] (N10) to node {}
node[swap] {} (N11);}
\only<23>{\draw<14->[red, ultra thick] [post] (N11) to node {}
node[swap] {} (N12);}
\end{tikzpicture}
\begin{columns}
\column[t]{.60\textwidth}
{\Large
\par\vspace{2cm}\noindent %Abstand zur Grafik 2cm
\begin{tabular}{l|lccccl}
\hline
Nr & Tätigkeit & Vorgänger & T$_{o}$ & T$_{w}$ & T$_{p}$ & Kurz\\
1 & Requirements & - & 1 & 2 & 3 & RQ \\
2 & Studie & - & 1 & 1 & 2 & Studie\\
3 & Sytsementwurf & 1 & 3 & 4 & 5 & SE\\
4 & 3 & 2 & - \\
5 & HW-Entwurf& 3 & 2 & 3 & 4 & HW\\
6 & Funktionsmuster & 3 & 1.5 & 2 & 2.5 & FM\\
7 & SW-Entwurf & 3 & 2.5 & 3 & 4 & SW\\
8 & Programmierung & 7 & 5 & 6 & 7.5 & Pgm \\
9 & 8 & 6 & - \\
10 & SW-Test & 8 & 4 & 5& 6 & SWT\\
11 & Prototyp-Entwicklung & 5 & 4 & 5 & 6 & Proto \\
12 & 11 & 6 & - & & \\
13 & HW-Test & 11 & 3 & 4 & 5 & HWT\\
14 & Integration & 10;13 & 1 & 2 & 3 & Int\\
15 & Sytsem-Test & 14 & 2 & 3 & 4 & Test\\
\hline
& \textbf{Gesamtaufwand} & & & \textbf{40} \\
\end{tabular}
}
\column[t]{.40\textwidth}
\par\vspace{1cm}\noindent %Abstand
\begin{itemize}
{\huge
\item<only@+> {Suche alle Vorgänge ohne Vorgänger (siehe Liste). Alle diese Pfeile gehen
von unserem ersten Knoten aus. Schreibe die Tätigkeiten (in Kurzform), sowie die fortlaufende
Nummer in den Knoten.}
\item<only@+> {Füge alle Pfeile zu den möglichen Nachfolger hinzu und schreibe T$_o$,
T$_w$ und T$_p$ über den Pfeil. Berechne anschließend T$_m$ und schreibe diesen unter den
Pfeil. } \\[1em]
\item<only@2> { T$_m$ = $\frac{ T_{o} + 4T_{w} + T_{p}}{6}$}
\item<only@+> {Wiederhole diesen Schritt.}
\item<only@+> {Wiederhole diesen Schritt.}
\item<only@+> {Wiederhole diesen Schritt.}
\item<only@+> {Wiederhole diesen Schritt.}
\item<only@+> {Wiederhole diesen Schritt.}
\item<only@+> {Dein Netzplan beginnt immer bei einem t$_F$ von 0. Schreibe dies über den
ersten Knoten}
\item<only@+> { Bei der Vorwärtsberechnung wird zum T$_{F}$ der T$_{m}$ dazugerechnet
und über den Knoten geschrieben. }
\item<only@+> {Wiederhole diesen Schritt.}
\item<only@+> {Wiederhole diesen Schritt.}
\item<only@+> {Wiederhole diesen Schritt.}
\item<only@+> {Wiederhole diesen Schritt.}
\item<only@+>{Bei einer Kreuzung wird bei der Vorwärtsberechnung mit dem größeren Wert
weitergearbeitet.}
\item<only@+>{Der Wert für T$_{S}$ wird von T$_{F}$ übernommen und unter den Knoten
geschrieben.}
\item<only@+>{Nun folgt die Rückwärtsberechnung. Hierbei wird T$_{m}$ von T$_{S}$
abgezogen.}
\item<only@+> {Bei der Rückwärtsberechnung wird T$_{m}$ von T$_{S}$ abgezogen und
dieser Wert wird unter den Knoten geschriieben.}
\item<only@+>{Bei einer Kreuzung wird bei der Rückwärtsberechnung mit dem kleineren
Wert weitergearbeitet.}
\item<only@+> {Die Differenz von T$_{m}$ und T$_{S}$ ergibt die Pufferzeit.}
\item<only@+> { Kritischer Pfad: T$_{S}$ - T$_{F}$ = 0 }
\item<only@+> {Zeichne den kritischen Pfad ein (hier: rot markiert).}
\item \alert<+> {}
}
\end{itemize}
\par\vspace{2cm}\noindent %Abstand
\only<23>{\begin{alertblock}{Merke!}
Alle Vorgänge mit einer Pufferzeit von 0 gehören zu dem kritischen Pfad.
\end{alertblock}
}
\end{columns}
\end{frame}
\end{document}