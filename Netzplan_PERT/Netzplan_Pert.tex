\documentclass{beamer}
\usepackage{german}
\usepackage{lmodern}
\usepackage{tikz}
%\usetikzlibrary{arrows,decorations.pathmorphing,backgrounds,positioning,fit,petri};
\usetikzlibrary{calc}
\makeatletter

\makeatother

\tikzset{
	pics/circle vertically split/.style 2 args = {
		code = {
			\node[inner sep=4pt,left] (-left) {#1};
			\node[inner sep=4pt,right] (-right) {#2};
			\path let
			\p1 = ($(-left.north west) - (-left.east)$),
			\p2 = ($(-right.west) - (-right.south east)$),
			\n1 = {max(veclen(\p1), veclen(\p2))*2}
			in node[minimum size=\n1, circle, draw] (-shape) at (0,0) {};
			\draw (-shape.north) -- (-shape.south);
		}
	}
}
		
\begin{document}
\begin{frame}[t,shrink=65]

\begin{tikzpicture}

\draw (0,0)  pic (01) {circle vertically split={Req}{1}}
(2,-4) pic (02) {circle vertically split={Stu}{2}}
(4,0)  pic (03) {circle vertically split={SE}{3}}
(8,4) pic (05) {circle vertically split={HWE}{5}}
(8,0) pic (06) {circle vertically split={Fun}{6}}
(8,-4) pic (07) {circle vertically split={SWE}{7}}
(12,-4) pic (08) {circle vertically split={Prog}{8}}
(16,-4) pic (10) {circle vertically split={SWT}{10}}
(12,4) pic (11) {circle vertically split={Prot}{11}}
(16,4) pic (13) {circle vertically split={HWT}{13}}
(20,0) pic (14) {circle vertically split={Int}{14}}
(24,0) pic (15) {circle vertically split={Sys}{15}};


\draw<2->[->, black, dashed]  (01-shape) -- (02-shape);
\draw<2->[->, black]  (01-shape) -- (03-shape) node[above,midway ]{1/2/3} node [below, midway]{2};
\draw<2->[->, black, dashed]  (02-shape) -- (03-shape);
\draw<3->[->, black]  (03-shape) -- (06-shape) node[above,midway ]{1.5/2/2.5};
\draw<3->[->, black]  (03-shape.north east) -- (05-shape) node[above,midway]{3/4/5};
\draw<3->[->, black]  (03-shape.south east) -- (07-shape) node[above,midway ]{2.5/3/4};
\draw<4->[->, black, dashed]  (06-shape.north east) -- (11-shape);
\draw<4->[->, black, dashed]  (06-shape.south east) -- (08-shape);
\draw<4->[->, black]  (05-shape) -- (11-shape) node[above,midway]{4/5/6};
\draw<4->[->, black]  (07-shape) -- (08-shape) node[above, midway]{5/6/7.5};
\draw<5->[->, black]  (11-shape) -- (13-shape) node[above,midway ]{3/4/5};
\draw<5->[->, black]  (08-shape) -- (10-shape) node[above,midway ]{4/5/6};
\draw<6->[->, black]  (13-shape) -- (14-shape);
\draw<6->[->, black]  (10-shape) -- (14-shape);
\draw<6->[->, black]  (14-shape) -- (15-shape) node[above,midway ]{2/3/4};


\end{tikzpicture}

\mode<presentation>
\definecolor{hdaRot}{cmyk}{0,1,1,0.18}     % -5005
\definecolor{hdaBlauMed}{cmyk}{0.71,0.25,0,0}     % -1169
\definecolor{hdaGruenMed}{cmyk}{0.5,0.13,1,0.06}    % -2048

{\Large
	\par\vspace{2cm}\noindent  
	
    \begin{tabular}{l|lcccl}
      \hline
    Nr & Tätigkeit & Vorgänger & T$_{o}$ & T$_{w}$ & T$_{p}$ \\
     1  & Requirements & - & 1 & 2 & 3  \\
     2  & Studie & - & 1 & 1 & 2  \\
     3  & Sytsementwurf & 1 & 3 & 4  & 5  \\
     4  & 3 & 2 & -  \\
     5  & HW-Entwurf& 3 & 2 & 3 & 4 \\
     6  & Funktionsmuster  & 3 & 1.5  & 2 & 2.5 \\
     7  & SW-Entwurf  & 3 & 2.5  & 3 & 4 \\
     8  & Programmierung & 7 & 5 & 6 & 7.5 \\
     9  & 8  & 6 & -  \\
    10  & SW-Test & 8 &  4 & 5& 6 \\
    11  & Prototyp-Entwicklung & 5 & 4 & 5 & 6  \\
	12  & 11  & 6 & - & & \\
	13  & HW-Test  & 11 & 3 & 4 & 5 \\
	14  & Integration & 10;13 & 1 & 2 & 3 \\
	15  & Sytsem-Test & 14 & 2 & 3 & 4 \\
      \hline
        & \textbf{Gesamtaufwand} & \textbf{40}  \\
    \end{tabular}
}

\par\vspace{1cm}\noindent 
\begin{itemize}
{\huge 
	\item<only@+> {Suche alle Vorgänge ohne Vorgänger). Alle diese Pfeile gehen vom ersten Knoten aus. Schreibe die Tätigkeiten in den Knoten, sowie Aufwand/Dauer auf den Pfeil.}
	
	\item<only@+> {Berechne außerdem noch T$_{m}$ und schreibe diese unter den Pfeil.}
	
	 \item<only@+> {Dein Netzplan beginnt immer bei einem t$_{F}$ von 0. Dieser wird über den Knoten geschrieben.}
	  
	  \item<only@+> {Bei der Vorwärtsrechnung wird zum T$_{F}$  der T$_{m}$ dazugerechnet.}
	  
	  \item<only@+> {Berechne T$_{F}$ für alle Knoten und schreibe diese über den jeweiligen Knoten.}
	  
	  \item<only@+>{Bei einer Kreuzung wird bei der Vorwärtsrechnung mit dem größeren Wert weitergearbeitet.}
	  
	  \item<only@+>{Der Wert für T$_{S}$ wird von  T$_{F}$ übernommen und unter den Knoten geschrieben.}
	  
	  \item<only@+>{Nun folgt die Rückwärtsrechnung. Hierbei wird T$_{m}$ von T$_{S}$ abgezogen.}
	  
	  \item<only@+>{Bei einer Kreuzung wird der kleinere Wert weiterverarbeitet.}
	  
	  \item<only@+>{Als letztes wird der kritische Pfad ermittelt und verdeutlicht.}
	  	
	  
	\item \alert<+> {}
}
\end{itemize}
	
\end{frame}

\end{document} 




alle knoten zeichnen 
beginnt bei TF = 0
werte überhalb des Pfeil aufschreiben 
wert unterhalb muss ausgerechnet werden und dann auch aufgeschrieben 
tF = tF + Tm
Vorwärtsrechnung: größeren Wert übernehmen 
Ts wird von Tf übernommen 
ts = ts -Tm
Rückwärtsrechnung: kleinerer Wert wird übernehmen 
kritischen Pfand einzeichnen: Puffer = 0 wenn ts-tf
